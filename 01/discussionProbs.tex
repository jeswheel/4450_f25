\documentclass[10pt,twoside]{article}
\usepackage[marginparsep=1em]{geometry}
\geometry{lmargin=1.0in,rmargin=1.0in, bmargin=1.2in,  tmargin=1.2in}
\usepackage[dvipsnames,svgnames,table]{xcolor}
\usepackage{graphicx}
\usepackage{amssymb}
\usepackage{epstopdf}
\usepackage{verbatim}
\usepackage{enumerate}
\usepackage{bm}
\usepackage{amsthm}
\usepackage{float}
\usepackage{amsmath}
\usepackage{fancyhdr}
\usepackage{hyperref}
\usepackage{mathtools}
            
%%%%  SHORTCUT COMMANDS  %%%%
\newcommand{\ds}{\displaystyle}
\newcommand{\Z}{\mathbb{Z}}
\newcommand{\T}{\mathcal{T}}
\newcommand{\arc}{\rightarrow}
\newcommand{\R}{\mathbb{R}}
\newcommand{\RP}{\mathbb{R}(+)}
\newcommand{\Rs}{\mathbb{R}^{**}}
\newcommand{\C}{\mathbb{C}}
\newcommand{\E}{\mathbb{E}}
\newcommand{\B}{\mathcal{B}}
\newcommand{\PX}{\mathcal{P}(X)}
\newcommand*\rot{\rotatebox{90}}
\newcommand*\OK{\ding{51}}
\newcommand{\N}{\mathbb{N}}
\newcommand{\Q}{\mathbb{Q}}
\newcommand{\Answer}{\vspace{2mm}\textbf{\underline{Answer}}\\}
\newcolumntype{L}{>{$}l<{$}}
\newcolumntype{C}{>{$}c<{$}}
\newcommand{\GL}{\text{GL}}
\newcommand{\SL}{\text{SL}}

\newcommand{\stirling}[2]{\genfrac{\{}{\}}{0pt}{}{#1}{#2}}

\pagestyle{fancy}
\fancyhf{}
\renewcommand{\sectionmark}[1]{\markright{\thesection.\ #1}}
\lhead{\fancyplain{}{}} 
\fancyhead[RE,RO]{Math 4450, Fall 2025}
\fancyfoot[RE,RO]{\thepage}


\begin{document}
\begin{flushright}
\begin{minipage}{.33\textwidth}
\rightline{Jesse Wheeler}
\rightline{\href{mailto:jeswheel@umich.edu}{jeswheel@umich.edu}}
\rightline{\today}
\end{minipage}
\end{flushright}

\begin{center}
{\large{\textbf{Chapter 1: Discussion Problems}}}
\end{center}

\begin{enumerate}
    
    % \item Seven balls are distributed randomly into seven cells. Let $X_i = $ the number of cells containing exactly $i$ balls. 
    % What is the probability distribution of $X_3$? (That is, find $P(X_3 = x)$ for every possible $x$.)
    % 
    % {\color{blue} Because there are only $7$ balls, the set of possible values of $X_3$ is the set $\mathcal{X}_3 = \{0, 1, 2\}$ (i.e. $X_3$ couldn't be greater than or equal to 3 because that would imply that there were at least $3\times 3 = 9$ balls).
    % 
    % Note that for each ball, there are 7 different bins into which a ball can be placed. Therefore there are $7^7 = 823,543$ total ways that the $7$ balls can be placed into $7$ bins. 
    % 
    % First consider how many ways can $X_3 = 2$. Because there are seven balls, the only outcomes such that $X_3 = 2$ occur when there are two bins with three balls, and one bin with one ball (the rest having no balls). There are $\binom{7}{2}$ different ways to choose which bins have $3$ balls, and $5$ different bins into which the last ball could be placed. In the first bin, there are $\binom{7}{3}$ ways of choosing which balls to place in the bin, leaving $\binom{4}{3}$ different ways of choosing which balls to plave into the second bin. This gives us $5\binom{7}{2}\binom{7}{3}\binom{4}{3} = 14,700$ possible ways of obtaining $X_3 = 2$.
    % 
    % Now consider how many different ways can $X_3 = 1$. This time, there are several different combinations of the number of balls in each bin, but using the same technique that was outlined in detail above, we get the following counts: 
    % \begin{align*}
    %     \text{1 bin with 3 balls, 1 bin with 4 balls: } 7\times6\times\binom{7}{3}&= 1470\\
    %     \text{1 bin with 3 balls, 4 bins with 1 ball: } 7 \times \binom{7}{3}\binom{6}{4}\binom{4}{1}\binom{3}{1}\binom{2}{1}&= 88200\\
    %     \text{1 bin with 3 balls, 1 bin with 2 balls, 2 bins with 1 ball: } 7\times 6\times \binom{7}{3}\binom{4}{2}\binom{5}{2}\binom{2}{1}&= 176400\\
    %     \text{1 bin with 3 balls, 2 bins with 2 balls: } 7\times \binom{7}{3}\binom{6}{2}\binom{4}{2} &= 22050 \\\hline
    %     \text{Total} &= 288,120
    % \end{align*}
    % 
    % Now note that we do not need to count the number of ways that $X_3 = 0$, because $\mathcal{X}_3 = \{0, 1, 2\}$ and therefore $P(X_3 = 0) = 1 - P(X_3 = 1) - P(X_3 = 2)$. 
    % 
    % Therefore the probability distribution of $X_3$ is given by:
    % \begin{align*}
    %     P_{X_3}(x) &= \begin{cases}  
    %         1 - \frac{288120}{823543} - \frac{14700}{823543} &X=0 \\
    %         \frac{288120}{823543} &X=1 \\
    %         \frac{14700}{823543} &X=2 
    %     \end{cases} \\
    %     &\approx \begin{cases}  
    %         0.632296 &X=0 \\
    %         0.349854 &X=1 \\
    %         0.01785 &X=2 
    %     \end{cases}
    % \end{align*}
    % }
    
    \item A closet contains $n$ pairs of shoes (that is, $2n$ total shoes). If $2r$ shoes are chosen at random $(2r < n)$, what is the probability that there will be no matching pair in the sample? 
    
    % {\color{blue} There are $2n$ shoes total and we are picking $2r$ of these shoes, so there are a total of $\binom{2n}{2r}$ different samples from which we could draw. 
    % 
    % Now I would like to find the number of samples in which there is no matching pair. First, there are $n$ total pairs to choose from. For a sample of $2r$ to have none of the pairs repeated, then there must be $2r$ unique pairs in that sample. Therefore there are $\binom{n}{2r}$ unique sets of $2r$ pairs of shoes. For each pair, I have two shoes to pick from. Therefore I have $2^{2r}$ different ways of picking which shoe within each chosen pair. Therefore there are $\binom{n}{2r}2^{2r}$ different samples of size $2r$ that do not contain a matching pair.
    % 
    % Therefore because each sample is equally as likely, the probability that there will be no matching pair in the sample is $\dfrac{\binom{n}{2r}2^{2r}}{\binom{2n}{2r}}$.}
    % 
    % \item Consider a multinomial experiment where at any given trial, $m$ distinct mutually exclusive and exhaustive outcomes can occur (e.g. a die is thrown and one of the six faces has to come up). Let $p_j$ be the probability that the $j$th outcome turns up in the trial. Then $\sum_{j = 1}^mp_j = 1$. Consider repeating the multinomial experiment independently (i.e. you keep throwing the die repeatedly) till $m_1, m_2, \ldots, m_k$ outcomes of types $1$ through $k$ (where $k < m)$ are obtained. What is the probability that you need $N$ trials of the experiment for this to happen? 
    
    % {\color{blue} For this to happen, the $N$th trial of the experiment must result in an outcome of type 1 through $k$ (otherwise we wouldn't have stopped). 
    % In order to make this problem a bit more simple, let's find the probability that the $N$th trial of the experiment resultsed in an outcome of type $j$, where $j \in \{1, \ldots, k\}$. 
    % For this to happen, we need outcomes of type $i$ to happen $m_i$ times, where $i \in \{1, \ldots, k\}$, 
    % and we need anything else (which has a probability of $1 - \sum_{i = 1}^kp_i$ of occurring) to happen $N - \sum_{i=1}^km_i$ different times. Therefore the probability for any given outcome of this type is given by $\left(\prod_{i=1}^kp_i^{m_i}\right)\left(1 - \sum_{i =1}^kp_i\right)^{N - \sum_{i=1}^km_i}$. For the sake of conciseness, let $p_{k+1} = \left(1 - \sum_{i =1}^kp_i\right)$ and let $m_{k+1} = N - \sum_{i=1}^km_i$. Therefore the probability of this result is:
    % 
    % $$
    % \left(\prod_{i=1}^kp_i^{m_i}\right)\left(p_{k+1}\right)^{m_{k+1}} = \left(\prod_{i=1}^{k+1}p_i^{m_i}\right)
    % $$
    % 
    % Now this event can happen in many different ways. Because we assumed that the $N$th trial resulted in an outcome of type $j$, there are a total of $N-1$ outcomes that we need to ``fill". First I will chose where to put the $m_1$ outcomes of type $1$, and then with the remaining spots I will chose where the $m_2$ outcomes of type $2$ go, and so forth. Therefore the following gives a count of the number of desired outcomes: 
    % \begin{align*}
    %     &\binom{N-1}{m_1}\binom{N-m_1-1}{m_2}\ldots\binom{N-\sum_{i =1}^{j-1}m_i - 1}{m_j - 1}\binom{N-\sum_{i =1}^{j}m_i}{m_{j+1}}\ldots\binom{N-\sum_{i =1}^{k-1}m_i}{m_{k}}\\
    %     &= \frac{(N-1)!(N-m_1-1)!\ldots\left(N-\sum_{i =1}^{j-1}m_i - 1\right)!\left(N-\sum_{i =1}^{j}m_i\right)!\ldots\left(N-\sum_{i =1}^{k-1}m_i\right)!}{(m_1!)(N-m_1-1)!(m_2!)(N-m_1-m_2-1)!\ldots(m_j-1)!\left(N-\sum_{i =1}^{j}m_i\right)!(m_{j+1})!\ldots \left(N-\sum_{i =1}^{k}m_i\right)!} \\
    %     &= \frac{(N-1)!}{m_1!m_2!\ldots(m_j-1)!\ldots m_k!(m_{k+1})!}\\
    %     &= \frac{(N-1)!}{(m_j - 1)! \times \prod_{i\neq j}^{k+1}\frac{1}{m_i!}}
    % \end{align*}
    % Therefore the probability of getting the desired counts in $N$ trials and having the final trial be of type $j$ is:
    % $$
    % \frac{(N-1)!}{(m_j - 1)! \times \prod_{i\neq j}^{k+1}\frac{1}{m_i!}}\times \left(\prod_{i=1}^{k+1}p_i^{m_i}\right)
    % $$
    % But now because the event of ending on type $j \in \{1, 2, \ldots, k\}$ is mutually exclusive to the event of ending on type $l \in \{1, 2, \ldots, k\}$ when $j \neq k$, then we can simply sum up the probabilities of these events to get the final answer: 
    % 
    % $$
    % \sum_{j = 1}^k\left[\frac{(N-1)!}{(m_j - 1)! \times \prod_{i\neq j}^{k+1}\frac{1}{m_i!}}\times \left(\prod_{i=1}^{k+1}p_i^{m_i}\right)\right]
    % $$
    % 
    % }
    

\end{enumerate}


\end{document}
