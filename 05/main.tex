\input{../header}

% \mode<beamer>{\usetheme{AnnArbor}}
\mode<beamer>{\usetheme{metropolis}}
\mode<beamer>{\metroset{block=fill}}
% \mode<beamer>{\usecolortheme{wolverine}}

\mode<beamer>{\setbeamertemplate{section in toc}[sections numbered]}
\mode<beamer>{\setbeamertemplate{subsection in toc}[subsections numbered indented]}

% \mode<beamer>{\usefonttheme{serif}}
\mode<beamer>{\setbeamertemplate{footline}}
\mode<beamer>{\setbeamertemplate{footline}[frame number]}
\mode<beamer>{\setbeamertemplate{frametitle continuation}[from second][\insertcontinuationcountroman]}
\mode<beamer>{\setbeamertemplate{navigation symbols}{}}

\mode<handout>{\pgfpagesuselayout{2 on 1}[letterpaper,border shrink=5mm]}

\newcommand\CHAPTER{5}
% \newcommand\answer[2]{\textcolor{blue}{#2}} % to show answers
% \newcommand\answer[2]{\textcolor{red}{#2}} % to show answers
 \newcommand\answer[2]{#1} % to show blank space

\title{\vspace{2mm} \link{https://jeswheel.github.io/4450_f25/}{Mathematical Statistics I}\\ \vspace{2mm}
Chapter \CHAPTER: Limit theorems}
\author{Jesse Wheeler}
\date{}

\setbeamertemplate{footline}[frame number]




\begin{document}

\maketitle

\mode<article>{\tableofcontents}

\mode<presentation>{
  \begin{frame}{Outline}
    \tableofcontents
  \end{frame}
}

\section{Convergence Concepts}

\begin{frame}{Introduction}
  \begin{itemize}
    \item This material comes primarily from \citet[][Chapter~4]{rice07}.
    \item We will cover the ideas of expected value, variance, as well has higher-order moments.
    \item This includes topics such as conditional expectation, which is one of the fundamental ideas behind many branches of statistics and machine learning.
    \item For instance, most regression / prediction algorithms are built with the idea of minimizing some conditional expectation.
  \end{itemize}
\end{frame}

% \begin{frame}[allowframebreaks]{}
% 
% \end{frame}

\newcommand\acknowledgments{
\begin{itemize}
\item   Compiled on {\today} using \Rlanguage version 4.5.1.
\item   \parbox[t]{0.75\textwidth}{Licensed under the \link{http://creativecommons.org/licenses/by-nc/4.0/}{Creative Commons Attribution-NonCommercial license}.
    Please share and remix non-commercially, mentioning its origin.}
    \parbox[c]{1.5cm}{\includegraphics[height=12pt]{../cc-by-nc}}
\item We acknowledge \link{https://jeswheel.github.io/4450_f25/acknowledge.html}{students and instructors for previous versions of this course / slides}.
\end{itemize}
}

\mode<presentation>{
\begin{frame}[allowframebreaks=0.8]{References and Acknowledgements}
  
\bibliography{../bib4450}

\vspace{3mm}

\acknowledgments

\end{frame}
}

\mode<article>{

\newpage

{\bf \Large \noindent Acknowledgments}

\acknowledgments

\newpage

\bibliography{../bib4450}

}



\end{document}







