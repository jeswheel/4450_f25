\documentclass[10pt,twoside]{article}\usepackage[]{graphicx}\usepackage[dvipsnames,svgnames,table]{xcolor}
% maxwidth is the original width if it is less than linewidth
% otherwise use linewidth (to make sure the graphics do not exceed the margin)
\makeatletter
\def\maxwidth{ %
  \ifdim\Gin@nat@width>\linewidth
    \linewidth
  \else
    \Gin@nat@width
  \fi
}
\makeatother

\definecolor{fgcolor}{rgb}{0.345, 0.345, 0.345}
\newcommand{\hlnum}[1]{\textcolor[rgb]{0.686,0.059,0.569}{#1}}%
\newcommand{\hlsng}[1]{\textcolor[rgb]{0.192,0.494,0.8}{#1}}%
\newcommand{\hlcom}[1]{\textcolor[rgb]{0.678,0.584,0.686}{\textit{#1}}}%
\newcommand{\hlopt}[1]{\textcolor[rgb]{0,0,0}{#1}}%
\newcommand{\hldef}[1]{\textcolor[rgb]{0.345,0.345,0.345}{#1}}%
\newcommand{\hlkwa}[1]{\textcolor[rgb]{0.161,0.373,0.58}{\textbf{#1}}}%
\newcommand{\hlkwb}[1]{\textcolor[rgb]{0.69,0.353,0.396}{#1}}%
\newcommand{\hlkwc}[1]{\textcolor[rgb]{0.333,0.667,0.333}{#1}}%
\newcommand{\hlkwd}[1]{\textcolor[rgb]{0.737,0.353,0.396}{\textbf{#1}}}%
\let\hlipl\hlkwb

\usepackage{framed}
\makeatletter
\newenvironment{kframe}{%
 \def\at@end@of@kframe{}%
 \ifinner\ifhmode%
  \def\at@end@of@kframe{\end{minipage}}%
  \begin{minipage}{\columnwidth}%
 \fi\fi%
 \def\FrameCommand##1{\hskip\@totalleftmargin \hskip-\fboxsep
 \colorbox{shadecolor}{##1}\hskip-\fboxsep
     % There is no \\@totalrightmargin, so:
     \hskip-\linewidth \hskip-\@totalleftmargin \hskip\columnwidth}%
 \MakeFramed {\advance\hsize-\width
   \@totalleftmargin\z@ \linewidth\hsize
   \@setminipage}}%
 {\par\unskip\endMakeFramed%
 \at@end@of@kframe}
\makeatother

\definecolor{shadecolor}{rgb}{.97, .97, .97}
\definecolor{messagecolor}{rgb}{0, 0, 0}
\definecolor{warningcolor}{rgb}{1, 0, 1}
\definecolor{errorcolor}{rgb}{1, 0, 0}
\newenvironment{knitrout}{}{} % an empty environment to be redefined in TeX

\usepackage{alltt}
\usepackage[marginparsep=1em]{geometry}
\geometry{lmargin=1.0in,rmargin=1.0in, bmargin=1.2in,  tmargin=1.2in}
\usepackage[dvipsnames,svgnames,table]{xcolor}
\usepackage{graphicx}
\usepackage{amssymb}
\usepackage{epstopdf}
\usepackage{verbatim}
\usepackage{enumerate}
\usepackage{bm}
\usepackage{amsthm}
\usepackage{float}
\usepackage{amsmath}
\usepackage{fancyhdr}
\usepackage{hyperref}
\usepackage{mathtools}
            
%%%%  SHORTCUT COMMANDS  %%%%
\newcommand{\ds}{\displaystyle}
\newcommand{\Z}{\mathbb{Z}}
\newcommand{\T}{\mathcal{T}}
\newcommand{\arc}{\rightarrow}
\newcommand{\R}{\mathbb{R}}
\newcommand{\RP}{\mathbb{R}(+)}
\newcommand{\Rs}{\mathbb{R}^{**}}
\newcommand{\C}{\mathbb{C}}
\newcommand{\E}{\mathbb{E}}
\newcommand{\B}{\mathcal{B}}
\newcommand{\PX}{\mathcal{P}(X)}
\newcommand*\rot{\rotatebox{90}}
\newcommand*\OK{\ding{51}}
\newcommand{\N}{\mathbb{N}}
\newcommand{\Q}{\mathbb{Q}}
\newcommand{\Answer}{\vspace{2mm}\textbf{\underline{Answer}}\\}
\newcolumntype{L}{>{$}l<{$}}
\newcolumntype{C}{>{$}c<{$}}
\newcommand{\GL}{\text{GL}}
\newcommand{\SL}{\text{SL}}

\newcommand{\stirling}[2]{\genfrac{\{}{\}}{0pt}{}{#1}{#2}}

\pagestyle{fancy}
\fancyhf{}
\renewcommand{\sectionmark}[1]{\markright{\thesection.\ #1}}
\lhead{\fancyplain{}{}} 
\fancyhead[RE,RO]{Math 4450, Fall 2025}
\fancyfoot[RE,RO]{\thepage}
\IfFileExists{upquote.sty}{\usepackage{upquote}}{}
\begin{document}
\begin{flushright}
\begin{minipage}{.33\textwidth}
\rightline{YOUR NAME}
\rightline{\href{mailto:YOUR EMAIL}{YOUR EMAIL}}
\rightline{\today}
\end{minipage}
\end{flushright}

\begin{center}
{\large{\textbf{Homework 4}}}
\end{center}

\begin{enumerate}
    
    \item Let $\alpha, \beta > 0$, and define the function $F(x)$ to be:
    $$
    F(x) = \begin{cases} 
      1 - e^{-\alpha x^\beta} & x \geq 0 \\
      0 & x < 0
    \end{cases}
    $$
    \begin{enumerate}
      \item Show that $F$ is a cdf for some random variable. (Hint: Theorem~2.1). 
      \item Calculate the corresponding pdf. 
    \end{enumerate}
    
    \item Let $\alpha \in [-1, 1]$, and define the function $f$ to be
    $$
    f(x) = \begin{cases}
      \frac{1+\alpha x}{2} & -1 \leq x \leq 1 \\
      0 & \text{otherwise}
    \end{cases}
    $$
    
    \begin{enumerate}
      \item Show that $f$ is a density for some continuous random variable. (Hint: Theorem 2.2. This theorem just gives properties of any pdf; the fact that it is tied to some random variable is a result of the next part of this question, and Theorem~2.1).
      \item Find the corresponding cdf. 
    \end{enumerate}
    
    \item For some value $c$, suppose that $X$ has the density $f(x) = cx^2$, with support $\mathcal{X} = [0, 1]$ (i.e., $f(x) = 0$ if for $x \notin [0, 1]$).
    \begin{enumerate}
      \item Find the value of $c$.
      \item What is $P(.1 \leq X < 0.5)$.
    \end{enumerate}
    
    \item If $X \sim N(0, \sigma^2)$, find the density of $Y = |X|$.
    \item Let $X \sim N(\mu, \sigma^2)$. Find the density of $Y = e^X$ (note: this is called the log-normal density, since $\log (Y) = X$ is normally distributed).
    \item If the radius of a circle is modeled as an exponential$(\lambda)$ random variable, find the pdf of the random variable $A$ that represents the area of the corresponding circle.

    % \item Let $X$ be a normally distributed random variable with mean $\mu$ and variance $\sigma^2$, i.e., $X \sim N(\mu, \sigma^2)$. The probability density function (PDF) of $X$ is given by:
    % $$
    % f_X(x) = \frac{1}{\sigma\sqrt{2\pi}}e^{\frac{1}{2\sigma^2}(x - \mu)^2}.
    % $$
    % The standard normal density function is defined as:
    % $$
    % \phi(z) = \frac{1}{\sqrt{2\pi}} e^{-x^2/2}.
    % $$
    % \textbf{Show that the PDF of} $X$ \textbf{can be written in terms of the standard normal density function} $\phi$ \textbf{as}:
    %   $$
    %   f_X(x) = \frac{1}{\sigma}\phi\Big(\frac{x-\mu}{\sigma}\Big).
    %   $$
    \item Theorem~2.3 from our lecture slides states that if $X$ is a random variable with CDF $F_X(x)$, and if $g$ is a strict, monotone function, then if $Y = g(X)$:
    \begin{enumerate}[(i)]
      \item \label{eq:increase} If $g$ is increasing, then $F_Y(y) = F_X\big(g^{-1}(y)\big)$,
      \item \label{eq:decreasing} If $g$ is decreasing, then $F_Y(y) = 1 - F_X\big(g^{-1}(y)\big)$.
    \end{enumerate}
    Part~(\ref{eq:decreasing}) was shown in class when $X$ is a \emph{continuous} random variable. Your task is the following:
    \begin{itemize}
      \item Assume that $X$ is a continuous random variable. 
      \textbf{Show that Theorem~2.3 holds when} $g$ \textbf{is a strictly monotonically increasing function}. 
      That is, show that the statement~(\ref{eq:increase}) is true.
      % \item Now assume that $X$ is a discrete random variable. \textbf{Show that Theorem~2.3 holds when} $g$ \textbf{is a strictly monotonically increasing function}. 
      % That is, show that the statement~(\ref{eq:increase}) is true.
    \end{itemize}
\end{enumerate}


\end{document}
