\documentclass[10pt,twoside]{article}\usepackage[]{graphicx}\usepackage[dvipsnames,svgnames,table]{xcolor}
% maxwidth is the original width if it is less than linewidth
% otherwise use linewidth (to make sure the graphics do not exceed the margin)
\makeatletter
\def\maxwidth{ %
  \ifdim\Gin@nat@width>\linewidth
    \linewidth
  \else
    \Gin@nat@width
  \fi
}
\makeatother

\definecolor{fgcolor}{rgb}{0.345, 0.345, 0.345}
\newcommand{\hlnum}[1]{\textcolor[rgb]{0.686,0.059,0.569}{#1}}%
\newcommand{\hlsng}[1]{\textcolor[rgb]{0.192,0.494,0.8}{#1}}%
\newcommand{\hlcom}[1]{\textcolor[rgb]{0.678,0.584,0.686}{\textit{#1}}}%
\newcommand{\hlopt}[1]{\textcolor[rgb]{0,0,0}{#1}}%
\newcommand{\hldef}[1]{\textcolor[rgb]{0.345,0.345,0.345}{#1}}%
\newcommand{\hlkwa}[1]{\textcolor[rgb]{0.161,0.373,0.58}{\textbf{#1}}}%
\newcommand{\hlkwb}[1]{\textcolor[rgb]{0.69,0.353,0.396}{#1}}%
\newcommand{\hlkwc}[1]{\textcolor[rgb]{0.333,0.667,0.333}{#1}}%
\newcommand{\hlkwd}[1]{\textcolor[rgb]{0.737,0.353,0.396}{\textbf{#1}}}%
\let\hlipl\hlkwb

\usepackage{framed}
\makeatletter
\newenvironment{kframe}{%
 \def\at@end@of@kframe{}%
 \ifinner\ifhmode%
  \def\at@end@of@kframe{\end{minipage}}%
  \begin{minipage}{\columnwidth}%
 \fi\fi%
 \def\FrameCommand##1{\hskip\@totalleftmargin \hskip-\fboxsep
 \colorbox{shadecolor}{##1}\hskip-\fboxsep
     % There is no \\@totalrightmargin, so:
     \hskip-\linewidth \hskip-\@totalleftmargin \hskip\columnwidth}%
 \MakeFramed {\advance\hsize-\width
   \@totalleftmargin\z@ \linewidth\hsize
   \@setminipage}}%
 {\par\unskip\endMakeFramed%
 \at@end@of@kframe}
\makeatother

\definecolor{shadecolor}{rgb}{.97, .97, .97}
\definecolor{messagecolor}{rgb}{0, 0, 0}
\definecolor{warningcolor}{rgb}{1, 0, 1}
\definecolor{errorcolor}{rgb}{1, 0, 0}
\newenvironment{knitrout}{}{} % an empty environment to be redefined in TeX

\usepackage{alltt}
\usepackage[marginparsep=1em]{geometry}
\geometry{lmargin=1.0in,rmargin=1.0in, bmargin=1.2in,  tmargin=1.2in}
\usepackage[dvipsnames,svgnames,table]{xcolor}
\usepackage{graphicx}
\usepackage{amssymb}
\usepackage{epstopdf}
\usepackage{verbatim}
\usepackage{enumerate}
\usepackage{bm}
\usepackage{amsthm}
\usepackage{float}
\usepackage{amsmath}
\usepackage{fancyhdr}
\usepackage{hyperref}
\usepackage{mathtools}
            
%%%%  SHORTCUT COMMANDS  %%%%
\newcommand{\ds}{\displaystyle}
\newcommand{\Z}{\mathbb{Z}}
\newcommand{\T}{\mathcal{T}}
\newcommand{\arc}{\rightarrow}
\newcommand{\R}{\mathbb{R}}
\newcommand{\RP}{\mathbb{R}(+)}
\newcommand{\Rs}{\mathbb{R}^{**}}
\newcommand{\C}{\mathbb{C}}
\newcommand{\E}{\mathbb{E}}
\newcommand{\B}{\mathcal{B}}
\newcommand{\PX}{\mathcal{P}(X)}
\newcommand*\rot{\rotatebox{90}}
\newcommand*\OK{\ding{51}}
\newcommand{\N}{\mathbb{N}}
\newcommand{\Q}{\mathbb{Q}}
\newcommand{\Answer}{\vspace{2mm}\textbf{\underline{Answer}}\\}
\newcolumntype{L}{>{$}l<{$}}
\newcolumntype{C}{>{$}c<{$}}
\newcommand{\GL}{\text{GL}}
\newcommand{\SL}{\text{SL}}
\newcommand{\floor}[1]{\lfloor #1 \rfloor}


\newcommand{\stirling}[2]{\genfrac{\{}{\}}{0pt}{}{#1}{#2}}

\pagestyle{fancy}
\fancyhf{}
\renewcommand{\sectionmark}[1]{\markright{\thesection.\ #1}}
\lhead{\fancyplain{}{}} 
\fancyhead[RE,RO]{Math 4450, Fall 2025}
\fancyfoot[RE,RO]{\thepage}
\IfFileExists{upquote.sty}{\usepackage{upquote}}{}
\begin{document}
\begin{flushright}
\begin{minipage}{.33\textwidth}
\rightline{YOUR NAME}
\rightline{\href{mailto:YOUR EMAIL}{YOUR EMAIL}}
\rightline{\today}
\end{minipage}
\end{flushright}

\begin{center}
{\large{\textbf{Homework 5}}}
\end{center}

\begin{enumerate}

    \item (1 point) Let $U \sim U(0, 1)$. For some $\lambda > 0$, let $g(x) = -\frac{1}{\lambda} \ln (1-x)$. What is the distribution of $Y = g(U)$?

    \item (1 point) Suppose that $X \sim N\big(1, 2^2\big)$ (i.e., $\sigma^2 = 4$). Find the density of $Y = X^3$.

    \item (2 points) Let $X \sim \text{Exp}(1)$, and $Y = (X - 1)^2$. Find the density function of $Y$. 

    \item (2 points) Consider the following experiment. An individual creates a standard cartesian X-Y plane (meaning we have a standard X-Y coordinate grid), and places their pencil at the point $(-1, 0)$. 
    Then, they draw a line towards the y-axis at a (uniform) random angle $\Theta$, such that $\Theta \sim U(-\pi/2, \pi/2)$.
    In doing so, their line will eventually cross the y-axis, at some random point $(0, Y)$. Find the pdf of the random variable $Y$, which represents where the line crosses the y-axis.
    
    \begin{itemize}
      \item (Hint): Draw a picture of what's happening. Try a few different ``random" lines from the point $(-1, 0)$ to the y-axis. Your solution will involve some Trigonometry. 
    \end{itemize}

    \item (Challenging) Let $X$ follow a Exponential$(\lambda)$ distribution, and denote $\floor{X}$ as the ``floor" of $X$, or the greatest integer not exceeding $X$ (e.g., $\floor{\pi} = 3$). 
    
    Now let $Y = X - \floor{X}$. Note that $Y$ is a continuous random variable, that takes on values in $[0, 1)$.
    \begin{enumerate}
      \item (1 point) Find the pmf of $\floor{X}$. 
        \begin{itemize}
          \item (Hint): Work from first principles, don't try the change of variables formula (though it works, just much harder). Because $\floor{X}$ is discrete, we want to find $P(\floor{X} = k)$ for interger values $k$. What does this mean for $X$?
          \item (LaTeX Hint): For those wanting to LaTeX, I had to create a new command \texttt{\textbackslash floor\{\}}. How this command is defined can be found in the header of the generating tex file.
        \end{itemize}
      \item (2 points) Find the pdf of $Y$. 
      \begin{itemize}
        \item (Hint): Consider finding $P(\floor{X} = m, Y \leq t)$, using a similar approach as above. Once that is done, then calculate
        $$P(Y \leq t) = \sum_{m} P(\floor{X} = m, Y \leq t)$$
      \end{itemize}
    \end{enumerate}
\end{enumerate}


\end{document}
