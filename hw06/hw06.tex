\documentclass[10pt,twoside]{article}\usepackage[]{graphicx}\usepackage[dvipsnames,svgnames,table]{xcolor}
% maxwidth is the original width if it is less than linewidth
% otherwise use linewidth (to make sure the graphics do not exceed the margin)
\makeatletter
\def\maxwidth{ %
  \ifdim\Gin@nat@width>\linewidth
    \linewidth
  \else
    \Gin@nat@width
  \fi
}
\makeatother

\definecolor{fgcolor}{rgb}{0.345, 0.345, 0.345}
\newcommand{\hlnum}[1]{\textcolor[rgb]{0.686,0.059,0.569}{#1}}%
\newcommand{\hlsng}[1]{\textcolor[rgb]{0.192,0.494,0.8}{#1}}%
\newcommand{\hlcom}[1]{\textcolor[rgb]{0.678,0.584,0.686}{\textit{#1}}}%
\newcommand{\hlopt}[1]{\textcolor[rgb]{0,0,0}{#1}}%
\newcommand{\hldef}[1]{\textcolor[rgb]{0.345,0.345,0.345}{#1}}%
\newcommand{\hlkwa}[1]{\textcolor[rgb]{0.161,0.373,0.58}{\textbf{#1}}}%
\newcommand{\hlkwb}[1]{\textcolor[rgb]{0.69,0.353,0.396}{#1}}%
\newcommand{\hlkwc}[1]{\textcolor[rgb]{0.333,0.667,0.333}{#1}}%
\newcommand{\hlkwd}[1]{\textcolor[rgb]{0.737,0.353,0.396}{\textbf{#1}}}%
\let\hlipl\hlkwb

\usepackage{framed}
\makeatletter
\newenvironment{kframe}{%
 \def\at@end@of@kframe{}%
 \ifinner\ifhmode%
  \def\at@end@of@kframe{\end{minipage}}%
  \begin{minipage}{\columnwidth}%
 \fi\fi%
 \def\FrameCommand##1{\hskip\@totalleftmargin \hskip-\fboxsep
 \colorbox{shadecolor}{##1}\hskip-\fboxsep
     % There is no \\@totalrightmargin, so:
     \hskip-\linewidth \hskip-\@totalleftmargin \hskip\columnwidth}%
 \MakeFramed {\advance\hsize-\width
   \@totalleftmargin\z@ \linewidth\hsize
   \@setminipage}}%
 {\par\unskip\endMakeFramed%
 \at@end@of@kframe}
\makeatother

\definecolor{shadecolor}{rgb}{.97, .97, .97}
\definecolor{messagecolor}{rgb}{0, 0, 0}
\definecolor{warningcolor}{rgb}{1, 0, 1}
\definecolor{errorcolor}{rgb}{1, 0, 0}
\newenvironment{knitrout}{}{} % an empty environment to be redefined in TeX

\usepackage{alltt}
\usepackage[marginparsep=1em]{geometry}
\geometry{lmargin=1.0in,rmargin=1.0in, bmargin=1.2in,  tmargin=1.2in}
\usepackage[dvipsnames,svgnames,table]{xcolor}
\usepackage{graphicx}
\usepackage{amssymb}
\usepackage{epstopdf}
\usepackage{verbatim}
\usepackage{enumerate}
\usepackage{bm}
\usepackage{amsthm}
\usepackage{float}
\usepackage{amsmath}
\usepackage{fancyhdr}
\usepackage{hyperref}
\usepackage{mathtools}
            
%%%%  SHORTCUT COMMANDS  %%%%
\newcommand{\ds}{\displaystyle}
\newcommand{\Z}{\mathbb{Z}}
\newcommand{\T}{\mathcal{T}}
\newcommand{\arc}{\rightarrow}
\newcommand{\R}{\mathbb{R}}
\newcommand{\RP}{\mathbb{R}(+)}
\newcommand{\Rs}{\mathbb{R}^{**}}
\newcommand{\C}{\mathbb{C}}
\newcommand{\E}{\mathbb{E}}
\newcommand{\B}{\mathcal{B}}
\newcommand{\PX}{\mathcal{P}(X)}
\newcommand*\rot{\rotatebox{90}}
\newcommand*\OK{\ding{51}}
\newcommand{\N}{\mathbb{N}}
\newcommand{\Q}{\mathbb{Q}}
\newcommand{\Answer}{\vspace{2mm}\textbf{\underline{Answer}}\\}
\newcolumntype{L}{>{$}l<{$}}
\newcolumntype{C}{>{$}c<{$}}
\newcommand{\GL}{\text{GL}}
\newcommand{\SL}{\text{SL}}
\newcommand{\floor}[1]{\lfloor #1 \rfloor}


\newcommand{\stirling}[2]{\genfrac{\{}{\}}{0pt}{}{#1}{#2}}

\pagestyle{fancy}
\fancyhf{}
\renewcommand{\sectionmark}[1]{\markright{\thesection.\ #1}}
\lhead{\fancyplain{}{}} 
\fancyhead[RE,RO]{Math 4450, Fall 2025}
\fancyfoot[RE,RO]{\thepage}
\IfFileExists{upquote.sty}{\usepackage{upquote}}{}
\begin{document}
\begin{flushright}
\begin{minipage}{.33\textwidth}
\rightline{YOUR NAME}
\rightline{\href{mailto:YOUR EMAIL}{YOUR EMAIL}}
\rightline{\today}
\end{minipage}
\end{flushright}

\begin{center}
{\large{\textbf{Homework 6}}}
\end{center}

\begin{enumerate}

    % \item Find the density of the minimum of $n$ independent Weibull random variables, each of which has the density
    % $$
    % f(t) = \beta \alpha^{-\beta} t^{\beta - 1}e^{-(t/\alpha)^\beta}
    % $$

    \item (1 point) If $X$ and $Y$ are independent exponential$(\lambda)$ random variables, find the joint density of the polar coordinates $R$ and $\Theta$ of the point $(X, Y)$. Are $R$ and $\Theta$ independent? 

    \item (1 point) Let $X$ and $Y$ be jointly continuous random variables, with pdf $f(x, y)$. Find an expression for the density of 
    $$
    Z = X - Y.
    $$

    \item (1 point) If $T_1$ and $T_2$ are independent exponential random variables, find the density function of
    $$
    R = T_{(2)} - T_{(1)}.
    $$

    \item (1 point) If $X_1$ and $X_2$ are independent Poisson random variables with parameters $\lambda_1$ and $\lambda_2$ respectively, then show that $S_2 = X_1 + X_2$ is a Poisson random variable with parameter $\lambda_1 + \lambda_2$.

    \item (1 point) Use the problem above and induction to argue that if $X_i$ for $i = 1, 2, \ldots, n$ are independent Poisson random variables with parameters $\lambda_i$, respectively, then if we define the sum as
    $$
    S_n = X_1 + X_2 + \ldots + X_n,
    $$
    then $S_n$ is a Poisson random variable with parameter $\sum_i \lambda_i$.
    

    \item Let $T$ be an exponential random variable with parameter $\beta$ and let $W$ be a random variable independent of $T$ which assumes the value $1$ with probability $2/3$ and the value $-1$ with probability $1/3$. 
    
    \begin{enumerate}
      \item (1 point) Find the density of $X = WT$.
      
    \textbf{Hint:} I suggest working from first-principles here. Consider $F(x) = P(X \leq x)$, and split up the event $\{X \leq x\}$ as the union of $\{X \leq x, W = 1\}$ and $\{X \leq x, W = -1\}$.
    
    \item (1 point) Find $E[X]$
    \end{enumerate}
    
    


    \item A random variable $V$ is said to have a distribution symmetric about $0$ if the distriubtion of $V$ is the same as that of $-V$. Let $V$ be a continuous random variable with continuous density function $f$.
    
    \begin{enumerate}
        \item (1 point) Show that $V$ is distributed symmetrically about $0$ if and only if $f(t) = f(-t)$, for every $t$. In other words, $f$ is an even function.
        
        \item (1 point) Show that $E(X) = 0$.
    \end{enumerate}
    
\end{enumerate}

\end{document}
