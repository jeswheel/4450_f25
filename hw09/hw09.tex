\documentclass[10pt,twoside]{article}\usepackage[]{graphicx}\usepackage[dvipsnames,svgnames,table]{xcolor}
% maxwidth is the original width if it is less than linewidth
% otherwise use linewidth (to make sure the graphics do not exceed the margin)
\makeatletter
\def\maxwidth{ %
  \ifdim\Gin@nat@width>\linewidth
    \linewidth
  \else
    \Gin@nat@width
  \fi
}
\makeatother

\definecolor{fgcolor}{rgb}{0.345, 0.345, 0.345}
\newcommand{\hlnum}[1]{\textcolor[rgb]{0.686,0.059,0.569}{#1}}%
\newcommand{\hlsng}[1]{\textcolor[rgb]{0.192,0.494,0.8}{#1}}%
\newcommand{\hlcom}[1]{\textcolor[rgb]{0.678,0.584,0.686}{\textit{#1}}}%
\newcommand{\hlopt}[1]{\textcolor[rgb]{0,0,0}{#1}}%
\newcommand{\hldef}[1]{\textcolor[rgb]{0.345,0.345,0.345}{#1}}%
\newcommand{\hlkwa}[1]{\textcolor[rgb]{0.161,0.373,0.58}{\textbf{#1}}}%
\newcommand{\hlkwb}[1]{\textcolor[rgb]{0.69,0.353,0.396}{#1}}%
\newcommand{\hlkwc}[1]{\textcolor[rgb]{0.333,0.667,0.333}{#1}}%
\newcommand{\hlkwd}[1]{\textcolor[rgb]{0.737,0.353,0.396}{\textbf{#1}}}%
\let\hlipl\hlkwb

\usepackage{framed}
\makeatletter
\newenvironment{kframe}{%
 \def\at@end@of@kframe{}%
 \ifinner\ifhmode%
  \def\at@end@of@kframe{\end{minipage}}%
  \begin{minipage}{\columnwidth}%
 \fi\fi%
 \def\FrameCommand##1{\hskip\@totalleftmargin \hskip-\fboxsep
 \colorbox{shadecolor}{##1}\hskip-\fboxsep
     % There is no \\@totalrightmargin, so:
     \hskip-\linewidth \hskip-\@totalleftmargin \hskip\columnwidth}%
 \MakeFramed {\advance\hsize-\width
   \@totalleftmargin\z@ \linewidth\hsize
   \@setminipage}}%
 {\par\unskip\endMakeFramed%
 \at@end@of@kframe}
\makeatother

\definecolor{shadecolor}{rgb}{.97, .97, .97}
\definecolor{messagecolor}{rgb}{0, 0, 0}
\definecolor{warningcolor}{rgb}{1, 0, 1}
\definecolor{errorcolor}{rgb}{1, 0, 0}
\newenvironment{knitrout}{}{} % an empty environment to be redefined in TeX

\usepackage{alltt}
\usepackage[marginparsep=1em]{geometry}
\geometry{lmargin=1.0in,rmargin=1.0in, bmargin=1.2in,  tmargin=1.2in}
\usepackage[dvipsnames,svgnames,table]{xcolor}
\usepackage{graphicx}
\usepackage{amssymb}
\usepackage{epstopdf}
\usepackage{verbatim}
\usepackage{enumerate}
\usepackage{bm}
\usepackage{amsthm}
\usepackage{float}
\usepackage{amsmath}
\usepackage{fancyhdr}
\usepackage{hyperref}
\usepackage{mathtools}
            
%%%%  SHORTCUT COMMANDS  %%%%
\newcommand{\ds}{\displaystyle}
\newcommand{\Z}{\mathbb{Z}}
\newcommand{\T}{\mathcal{T}}
\newcommand{\arc}{\rightarrow}
\newcommand{\R}{\mathbb{R}}
\newcommand{\RP}{\mathbb{R}(+)}
\newcommand{\Rs}{\mathbb{R}^{**}}
\newcommand{\C}{\mathbb{C}}
\newcommand{\E}{\mathbb{E}}
\newcommand{\B}{\mathcal{B}}
\newcommand{\PX}{\mathcal{P}(X)}
\newcommand*\rot{\rotatebox{90}}
\newcommand*\OK{\ding{51}}
\newcommand{\N}{\mathbb{N}}
\newcommand{\Q}{\mathbb{Q}}
\newcommand{\Answer}{\vspace{2mm}\textbf{\underline{Answer}}\\}
\newcolumntype{L}{>{$}l<{$}}
\newcolumntype{C}{>{$}c<{$}}
\newcommand{\GL}{\text{GL}}
\newcommand{\SL}{\text{SL}}
\newcommand{\floor}[1]{\lfloor #1 \rfloor}
\newcommand\var[1]{\mathrm{Var}\left[{#1}\right]}
\newcommand\myvar{\mathrm{Var}}
\newcommand\Var{\myvar}
\newcommand\cov{\mathrm{Cov}}
\newcommand\Cov{\mathrm{Cov}}
\newcommand\plim{\overset{p}{\rightarrow}}
\newcommand\Plim{\overset{P}{\rightarrow}}
\newcommand\dlim{\overset{d}{\rightarrow}}
\newcommand\Dlim{\overset{D}{\rightarrow}}


\newcommand{\stirling}[2]{\genfrac{\{}{\}}{0pt}{}{#1}{#2}}

\pagestyle{fancy}
\fancyhf{}
\renewcommand{\sectionmark}[1]{\markright{\thesection.\ #1}}
\lhead{\fancyplain{}{}} 
\fancyhead[RE,RO]{Math 4450, Fall 2025}
\fancyfoot[RE,RO]{\thepage}
\IfFileExists{upquote.sty}{\usepackage{upquote}}{}
\begin{document}
\begin{flushright}
\begin{minipage}{.33\textwidth}
\rightline{YOUR NAME}
\rightline{\href{mailto:YOUR EMAIL}{YOUR EMAIL}}
\rightline{\today}
\end{minipage}
\end{flushright}

\begin{center}
{\large{\textbf{Homework 9 (Chapter 5)}}}
\end{center}


\begin{enumerate}

  \item (1 point) Find the moment generating function for the density $f(x) = 2x/c^2, 0 < x < c$. 
  \item (1 point) Let $X_1, X_2, \ldots$ be a sequence of independent random variables with $E(X_i) = \mu$, and $\Var(X_i) = \sigma^2_i$. Show that if $n^{-2}\sum_i \sigma^2_i \rightarrow 0$, then $\bar{X} \plim \mu$ (This is a statement in convergence in probability. Check Chapter~5 notes for a definition, as well as the examples for how you might solve this problem).
  \item (1 point) Let $X_1, X_2, \ldots$ be as in the problem above, but with $E(X_i) = \mu_i$, and $n^{-1}\sum_{i = 1}^n \mu_i \rightarrow \mu$. Show that $\bar{X} \plim \mu$. \textbf{Hint:} you might need to use the triangle inequality.
  \item Let $X$ follow a Binomial$(n, p)$ distribution.
  \begin{enumerate}
    \item (1 point) Using the Binomial Theorem (see, for instance, Proposition 1.2 slides from Chapter 1), derive the MGF of $X$.
    \item (1 points) Using moment-generating functions, show that as $n\rightarrow \infty$, $p\rightarrow 0$, and $np \rightarrow \lambda$, 
  the binomial distribution with parameters $n$ and $p$ tends to the Poisson distribution.
  \end{enumerate}
  \item (1 point) Using moment-generating functions, show that as $\alpha \rightarrow \infty$, the gamma distribution with parameters $\alpha$ and $\lambda$, properly standardized, tends to the standard normal distribution. (\textbf{Hint}: we already calculated the MGF of a Gamma distribution in class. No need to rederive it here.)
  \item (1 point) Suppose that $X_1, \ldots, X_{20}$ are independent random variables with density functions
    $$
    f(x) = 2x, \quad 0\leq x\leq 1.
    $$
    Let $S = X_1 + \ldots + X_{20}$. Use the central limit theorem to approximate $P(S \leq 10)$.
  \item (2 points) Suppose that a measurement $X_i$ has mean $\mu$ and variance $\sigma^2 = 25$. Let $\bar{X}$ be the average of $n$ such independent measurements.
  How large should $n$ be so that $P\big(|\bar{X} - \mu| < 1\big) = 0.95$? 
  \begin{itemize}
    \item Note that there are various approaches to this problem. Two we have covered in this class, namely using the CLT to approximate the distribution, or using Chebyshev's inequality. Try both approaches, and comment on the results.
  \end{itemize}
  
\end{enumerate}

\end{document}
